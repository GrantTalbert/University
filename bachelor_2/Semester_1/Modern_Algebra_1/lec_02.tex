\lecture{2}{Thu 05 Sep 2024 09:30}{Review of Proofs}

\noindent No discussion tomorrow - and they will all be canceled until the grad students get tf off strike

\begin{notation}
	For $a,b\in\mathbb{Z}$, if $a$ divides $b$, that is $b/a\in\mathbb{Z}$, then we weite $a|b$ to mean $a$ divides $b$.
\end{notation}
\section{Integers Mod $n$}
\begin{definition}[Integer Equivalence mod n]\label{dfn:3}
	Integers $a,b\in\mathbb{Z}$ are equivalent mod $n$ if $n$ divides $a-b$ (the remainders are the same), and we write
	\[
		a\equiv b\;(mod\,n)
	.\]
\end{definition}
For example, $7+4\equiv 1\;(mod\,5)$, since $5|(11-1)$.
\begin{definition}[Integers modulo $n$]\label{dfn:4}
	The set of integers modulo $n$ is the set $\left\{ 0,\ldots,n-1 \right\}$, and is denoted $\mathbb{Z}_n$. In $\mathbb{Z}_n$, addition and multiplication are done modulo $n$.
\end{definition}
\begin{center}
	The Cayley Table for $\mathbb{Z}_6$ \\
	\begin{tabular}{c|cccccc}
		+&0&1&2&3&4&5\\
		\hline
		0&0&1&2&3&4&5\\
		1&1&2&3&4&5&0\\
		2&2&3&4&5&0&1\\
		3&3&4&5&0&1&2\\
		4&4&5&0&1&2&3\\
		5&5&0&1&2&3&4
	\end{tabular}
\end{center}
\begin{proposition}
	Let $\mathbb{Z}_n$ be the set of integers modulo $n$, and let $a,b,c\in\mathbb{Z}_n$. We have
	\begin{itemize}
		\item $a+(b+c)=(a+b)+c\mod n$.
		\item  There exists an additive identity  $0$ such that for all $a\in\mathbb{Z}_n,$ $a+0=a\mod n$.
		\item For every $a\in\mathbb{Z}_n$, there exists an additive inverse $-a\in\mathbb{Z}_n$ such that $a+(-a)=-a+a=0\mod n$.
		\item $a+b=b+a\mod n$ for all $a,b\in\mathbb{Z}$.
	\end{itemize}
\end{proposition}
\begin{proof}
	(1) Since $(a+b)+c=a+(b+c)$ in the integers, then the remainders mod $n$ are also equal.\\
	The rest of the proof is left as an exercise
\end{proof}
\chapter{Groups}
\begin{definition}[Binary Operation]\label{dfn:5}
	Let $G$ be a set. A binary operation on $G$ is a function that assigns each ordered pair of elements of $G$ to an element of $G$.
	\[
		\cdot : G\times G\to G
	.\]
\end{definition}
For example, in the case of $G=D_4$, then function composition $\circ (A,B)=BA$ is a binary operation on $G$. If $G=\mathbb{Z}_n$, then the binary operation is addition $+(a,b)=a+b\mod n$.
\begin{definition}[Group]\label{dfn:6}
	Let $G$ be a set together with a binary operation under which $G$ is closed:
	\[
		\cdot : G\times G \to G
	\]
	\[
		\cdot : (a,b) \mapsto ab
	.\]
We say that $G$ is a group under this operation if the following properties are satisfied:
\begin{enumerate}
	\item Associativity - for any $a,b,c\in G$, $a(bc)=(ab)c$.
	\item Identity - there exists some $e\in G$ such that for all $g\in G$, $ge=eg=g$.
	\item Inverses - for all $a\in G$, there exists a corresponding $b\in G$ such that $ab=ba=e$. This is usually denoted $a^{-1}$.
\end{enumerate}
\end{definition}
\begin{definition}[Abelian Group]\label{dfn:7}
	Let $G$ be a group. We call $G$ an \textbf{abelian} group if $ab=ba$ for all $a,b\in G$ (commutative property). Otherwise, the group is non-abelian.
\end{definition}

For example, $D_4$ under function composition is called the Dihedral group of order 8, and $\mathbb{Z}_n$ under addition mod $n$ is the group of integers mod $n$. $D_4$ is non-abelian, while $\mathbb{Z}_n$ is abelian. More examples:
\begin{itemize}
	\item $\mathbb{Z}$ under addition is a group.
	\item $\mathbb{Z}$ under division is \textbf{not} a group.
	\item $\mathbb{Z}$ under multiplication is \textbf{not} a group.
	\item $\mathbb{R}^*$ (the set of nonzero reals) is a group under multiplication.
	\item $M_2(\mathbb{R})$ (set of $2\times 2$ matrices with real entries) is a group under addition.
	\item $\operatorname{GL}_2(\mathbb{R})\subseteq M_2(\mathbb{R}) $ the general linear is a group under multiplication.
\end{itemize}
QUATERNIONS!

Let $1$ be the identity matrix,
 \[
	 I=\begin{pmatrix}0&1\\-1&0\end{pmatrix},\quad J=\begin{pmatrix}0&i\\i&0\end{pmatrix},\quad K=\begin{pmatrix}i&0\\0&-i\end{pmatrix}
,\]
where $i^2=-1$. Then
\[
	I^2=J^2=K^2=-1,\quad IJ=K,\quad JK=-I,\quad KI=J,\quad JI=-K,
\]
\[
	KJ=-I,\quad IK=-J
.\]
The group $\left\{ \pm 1, \pm I, \pm J, \pm K \right\} $ is knpwn as the quaternion group under multipliaction.


