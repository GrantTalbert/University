\documentclass[11pt, letterpaper]{report}
%basic packages
\usepackage[utf8]{inputenc}
\usepackage[T1]{fontenc}
\usepackage{graphicx}
\usepackage[margin=0.75in]{geometry}
\usepackage[usenames,dvipsnames]{xcolor}

%math
\usepackage{amsmath, amsthm, amsfonts, amssymb, mathtools}
\usepackage{mathrsfs}
\usepackage{cancel}
\usepackage{siunitx} %phyjsicsssss
\usepackage{bbm} %mathbb for numbers
\usepackage[all]{xy} % https://texdoc.org/serve/xyguide.pdf/0
\makeatletter
\renewcommand*\env@matrix[1][c]{\hskip -\arraycolsep
  \let\@ifnextchar\new@ifnextchar
  \array{*\c@MaxMatrixCols #1}}
\makeatother %matrix realignment

%misc
\usepackage{float}
\usepackage[hyphens]{url}
%\definecolor{page}{HTML}{242526}
%\pagecolor{page}
\usepackage{booktabs} %the \toprule and \bottomrule thick lines on tables

\usepackage{hyperref}
\definecolor{aqua}{HTML}{00C5FF}
\hypersetup{
    colorlinks,
    linkcolor={aqua},
    urlcolor={aqua},
    citecolor={red}
}

%my commands
\DeclarePairedDelimiter\bra{\langle}{\rvert} %Bra
\DeclarePairedDelimiter\ket{\lvert}{\rangle} %Ket
\DeclarePairedDelimiterX\braket[2]{\langle}{\rangle}{#1\,\delimsize\vert\,\mathopen{}#2} %Bra-ket
\newcommand{\pvec}[1]{\vec{#1}\mkern2mu\vphantom{#1}} % from https://tex.stackexchange.com/questions/120029/how-to-typeset-a-primed-vector
\newcommand{\hati}{\boldsymbol{\hat{\textbf{\i}}}}
\newcommand{\hatj}{\boldsymbol{\hat{\textbf{\j}}}}
\newcommand{\hatk}{\boldsymbol{\hat{\textbf{k}}}}
\newcommand{\R}{\mathbb{R}}
\DeclareMathOperator{\diag}{diag}

%theorems
\usepackage{thmtools}
\usepackage{tikz}
\usepackage{tikz-cd}
\usepackage[framemethod=TikZ]{mdframed}
\mdfsetup{skipabove=1em,skipbelow=0em, innertopmargin=5pt, innerbottommargin=6pt}
\theoremstyle{definition} %because obviously
% THEOREM STYLES

\definecolor{boxColor}{HTML}{3D3D3D}

% Definitions
\newcounter{def}[chapter]\setcounter{def}{0}
\renewcommand{\thedef}{\arabic{chapter}.\arabic{def}}
\newenvironment{definition}[1][]{
  \refstepcounter{def}
  \ifstrempty{#1}{
    \mdfsetup{
      frametitle = {
        \tikz[baseline=(current bounding box.east),outer sep=0pt]
        \node[anchor=east,rectangle,font=\bfseries,fill=white]{\strut Definition~\thedef};}}}
  {\mdfsetup{
      frametitle={
        \tikz[baseline=(current bounding box.east),outer sep=0pt]
        \node[anchor=east,rectangle,font=\bfseries,fill=white]{\strut Definition~\thedef :~#1};}}}
  \mdfsetup{innertopmargin=-5pt,linewidth=1pt,topline=true,nobreak,frametitleaboveskip=\dimexpr-12pt\relax\strutbox}
\begin{mdframed}[]}{\end{mdframed}}

% Theorems
\newcounter{thm}[chapter]\setcounter{thm}{0}
\renewcommand{\thethm}{\arabic{chapter}.\arabic{thm}}
\newenvironment{theorem}[1][]{
  \refstepcounter{thm}
  \ifstrempty{#1}{
    \mdfsetup{
      frametitle = {
        \tikz[baseline=(current bounding box.east),outer sep=0pt]
        \node[anchor=east,rectangle,font=\bfseries,fill=white,draw,thick]{\strut Theorem~\thethm};}}}
  {\mdfsetup{
      frametitle={
        \tikz[baseline=(current bounding box.east),outer sep=0pt]
        \node[anchor=east,rectangle,font=\bfseries,fill=white,draw,thick]{\strut Theorem~\thethm~(#1)};}}}
  \mdfsetup{innertopmargin=2pt,linewidth=1pt,topline=true,nobreak,frametitleaboveskip=\dimexpr-13pt\relax\strutbox}
\begin{mdframed}[]}{\end{mdframed}}

% Lemmas
\newcounter{lma}[chapter]\setcounter{lma}{0}
\renewcommand{\thelma}{\arabic{chapter}.\arabic{lma}}
\newenvironment{lemma}[1][]{
  \refstepcounter{lma}
  \ifstrempty{#1}{
    \mdfsetup{
      frametitle = {
        \tikz[baseline=(current bounding box.east),outer sep=0pt]
        \node[anchor=east,rectangle,font=\bfseries,fill=white,draw,thick]{\strut Lemma~\thelma};}}}
  {\mdfsetup{
      frametitle={
        \tikz[baseline=(current bounding box.east),outer sep=0pt]
        \node[anchor=east,rectangle,font=\bfseries,fill=white,draw,thick]{\strut Lemma~\thelma~(#1)};}}}
  \mdfsetup{innertopmargin=2pt,linewidth=1pt,topline=true,nobreak,frametitleaboveskip=\dimexpr-13pt\relax\strutbox}
\begin{mdframed}[]}{\end{mdframed}}


% Corollaries
\newcounter{corr}[chapter]\setcounter{corr}{0}
\renewcommand{\thecorr}{\arabic{chapter}.\arabic{corr}}
\newenvironment{corollary}[1][]{
  \refstepcounter{corr}
  \ifstrempty{#1}{
    \mdfsetup{
      frametitle = {
        \tikz[baseline=(current bounding box.east),outer sep=0pt]
        \node[anchor=east,rectangle,font=\bfseries,fill=white,draw,thick]{\strut Corollary~\thecorr};}}}
  {\mdfsetup{
      frametitle={
        \tikz[baseline=(current bounding box.east),outer sep=0pt]
        \node[anchor=east,rectangle,font=\bfseries,fill=white,draw,thick]{\strut Corollary~\thecorr~(#1)};}}}
  \mdfsetup{innertopmargin=2pt,linewidth=1pt,topline=true,nobreak,frametitleaboveskip=\dimexpr-13pt\relax\strutbox}
\begin{mdframed}[]}{\end{mdframed}}

% Propositions
\newcounter{prp}[chapter]\setcounter{prp}{0}
\renewcommand{\theprp}{\arabic{chapter}.\arabic{prp}}
\newenvironment{proposition}[1][]{
  \refstepcounter{prp}
  \ifstrempty{#1}{
    \mdfsetup{
      frametitle = {
        \tikz[baseline=(current bounding box.east),outer sep=0pt]
        \node[anchor=east,rectangle,font=\bfseries,fill=white,draw,thick]{\strut Proposition~\theprp};}}}
  {\mdfsetup{
      frametitle={
        \tikz[baseline=(current bounding box.east),outer sep=0pt]
        \node[anchor=east,rectangle,font=\bfseries,fill=white,draw,thick]{\strut Proposition~\theprp~(#1)};}}}
  \mdfsetup{innertopmargin=2pt,linewidth=1pt,topline=true,nobreak,frametitleaboveskip=\dimexpr-13pt\relax\strutbox}
\begin{mdframed}[]}{\end{mdframed}}

% Remarks
\newenvironment{remark}[1][]{
  \mdfsetup{
      frametitle={
        \tikz[baseline=(current bounding box.east),outer sep=0pt]
        \node[anchor=east,rectangle,font=\bfseries,fill=white]{\strut Remark};}}
  \mdfsetup{innertopmargin=-7pt,linewidth=1pt,topline=true,bottomline=true,leftline=false,rightline=false,nobreak,frametitleaboveskip=\dimexpr-12pt\relax\strutbox}
\begin{mdframed}[]}{\end{mdframed}}

% Examples
\newenvironment{example}[1][]{
  \mdfsetup{
      frametitle={
        \tikz[baseline=(current bounding box.east),outer sep=0pt]
        \node[anchor=east,rectangle,font=\bfseries,fill=white,draw,thick]{\strut Example};}}
  \mdfsetup{innertopmargin=0pt,linewidth=1pt,topline=true,bottomline=true,leftline=false,rightline=false,nobreak,frametitleaboveskip=\dimexpr-13pt\relax\strutbox}
\begin{mdframed}[]}{\end{mdframed}}

% As Previously Seen
\newenvironment{prev}[1][]{
  \mdfsetup{
      frametitle={
        \tikz[baseline=(current bounding box.east),outer sep=0pt]
        \node[anchor=east,rectangle,font=\bfseries,fill=white]{\strut As Previously Seen};}}
  \mdfsetup{innertopmargin=-7pt,linewidth=1pt,topline=true,bottomline=true,leftline=false,rightline=false,nobreak,frametitleaboveskip=\dimexpr-12pt\relax\strutbox}
\begin{mdframed}[]}{\end{mdframed}}

% Exercises
\newcounter{exe}[chapter]\setcounter{exe}{0}
\renewcommand{\theexe}{\arabic{chapter}.\arabic{exe}}
\newenvironment{exercise}[1][]{
  \refstepcounter{exe}
  \mdfsetup{
      frametitle={
        \tikz[baseline=(current bounding box.east),outer sep=0pt]
        \node[anchor=east,rectangle,font=\bfseries,fill=white,draw,thick]{\strut Exercise~\theexe};}}
  \mdfsetup{roundcorner=10pt,innertopmargin=0pt,linewidth=1pt,topline=true,nobreak,frametitleaboveskip=\dimexpr-13pt\relax\strutbox}
\begin{mdframed}[]}{\end{mdframed}}

%theorem styles
\declaretheoremstyle[headfont=\bfseries, bodyfont=\normalfont, mdframed={linewidth=1pt,   bottomline=false, topline=false, rightline=false}, qed=\(\blacksquare\)]{proofline}
\declaretheoremstyle[headfont=\bfseries, bodyfont=\normalfont, mdframed={linewidth=1pt,   bottomline=false, topline=false, rightline=false}, qed=\qedsymbol]{egline}

\declaretheorem[numbered=no, name=Notation]{notation}

%solution environment
\declaretheorem[numbered=no, style=egline, name=Solution]{setsolution}
\newenvironment{solution}[1][]{\vspace{-10pt}\begin{setsolution}}{\end{setsolution}}
%proof environment
\declaretheorem[numbered=no, style=proofline, name=Proof]{replacementproof}
\renewenvironment{proof}[1][\proofname]{\begin{replacementproof}}{\end{replacementproof}}

% Side Indented Theorems - https://tex.stackexchange.com/questions/429339/shifting-newtheorem
\newtheoremstyle{side}{}{}{\advance\leftskip3cm\relax\itshape\normalfont}{-4pt}
{\bfseries}{}{0pt}{
\makebox[0pt][r]{
  \smash{\parbox[t]{2.5cm}{\raggedright\thmname{#1}.
  \thmnote{\newline(#3)}}}
  \hspace{10.1pt}}}

\theoremstyle{side}
\newtheorem{note}{Note}
\newtheorem{intuition}{Intuition}
\newtheorem{claim}{Claim}
\theoremstyle{definition}

%pulls lecture files
\newcommand{\lec}[2]{%
	\foreach \c in {#1,...,#2}{%
		\IfFileExists{Lectures/lec_\c.tex} {%
			\input{Lectures/lec_\c.tex}%
		}{}%
	}%
}
%to use, in the same file directory as your header.tex and master.tex files, create a folder titled "Lectures" and put your
%lectures into their, named lec_1.tex, lec_2.tex, and so on. In the master.tex file, write \lec{a}{b} where a is the lowest
%number you want to call, and b is the highest.

%fancy headers
\usepackage{fancyhdr}
\pagestyle{fancy}
\fancyhead{}\fancyfoot{}
\fancyfoot[R]{\thepage}
\fancyfoot[C]{\leftmark}


%figures, taken from (https://castel.dev/post/lecture-notes-2)
\usepackage{import}
\usepackage{xifthen}
\usepackage{pdfpages}
\usepackage{transparent}
\newcommand{\incfig}[1]{%
    \def\svgwidth{\columnwidth}
    \import{./figures/}{#1.pdf_tex}
}

%lectures, taken from (https://castel.dev/post/lecture-notes-3)
\makeatother
\def\@lecture{}%
\newcommand{\lecture}[3]{%
	\ifthenelse{\isempty{#3}}{%

		\def\@lecture{Lecture #1}%
	}{%
		\def\@lecture{Lecture #1: #3}
	}
	\subsection*{\@lecture}
	\hfill{\small\textsf{#2}}\par
}
\makeatletter

\author{Grant Talbert}

\usepackage{titlepageBU}
\title{Modern Algebra 1}
\date{09/26/24}
\courseID{MA 541}
\professor{Dr. Duque-Rosero}
\courseSection{A1}
\renewcommand{\mod}[1]{\;(\text{mod}\;#1)}
\begin{document}
\makeproblem
\section*{Problem 1}
\begin{solution}
	For convenience, let $I$ always denote the $2\times 2$ identity matrix within this problem, and let $H\coloneqq \left\{ \alpha I  \mid \alpha \in\mathbb{Z}_n \land \alpha ^2\not\equiv0\mod{n}\right\} $.

	First, we show that $H\subseteq Z(\operatorname{GL}_2(\mathbb{Z}_n)) $. Let $A \in \operatorname{GL}_2(\mathbb{Z}_n) $, and let $\alpha \in\mathbb{Z}_n$. It follows that
	\[
		(\alpha I)A=\alpha (IA)=\alpha A=A\alpha =(AI)\alpha =A(I\alpha )=A(\alpha I)
	.\]
	Thus, for all $A\in \operatorname{GL}_2(\mathbb{Z}_n) $, $\alpha IA=A\alpha I$. Therefore, $H\subseteq Z(\operatorname{GL}_2(\mathbb{Z}_n)) $.

	Before we show the converse, I have some commentary on the problem. The first line of the problem states $\operatorname{GL}_2(\mathbb{Z}_n) $ is a group for $n\geq 2$. After a lot of confusion, it turns out $\operatorname{GL}_2(\mathbb{Z}_n) $ is only a group for $n$ a prime integer. This is fairly easy to prove. Let $A,B\in \operatorname{GL}_2(\mathbb{Z}_n) $. Then $\operatorname{GL}_2(\mathbb{Z}_n) $ is a group if and only if $AB\in \operatorname{GL}_2(\mathbb{Z}_n) $. I leae it as an exercise to prove the other requirements for a group, but they will be satisfied for any $n$. For any integer $0<a<n$, we can very easily construct a matrix $A$ such that $\left| A \right| =a$. Let $0<a,b<n$, and let $\left| A \right| =a$ an $\left| B \right| =b$. We thus have $\det (AB)=ab$. Therefore, $AB\in \operatorname{GL}_2(\mathbb{Z}_n) $ if and only if $ab\not\equiv 0\mod{n}$. For any $n$ not prime, there exist $a,b\in\mathbb{Z}$ such that this statement is false. Therefore, $n$ must be prime. The conclusion of all this is that the set being over $\mathbb{Z}_n$ literally doesn't matter and this should be provable via normal matrix algebra.

Now, we show the converse. 
\end{solution}
\section*{Problem 2}
\begin{solution}
	This was easy, in fact we did an example of this last homework. Let $G=\left\{ e,a,b,c \right\} $ be a group with identity $e$. Let $a^2=b^2=c^2=e$, and let $ab=ba=c$. It follows that $bc=cb=a$ and $ac=ca=b$. We give the following Cayley table to help with this visualization.
	\begin{center}
	\begin{tabular}{c|cccc}
		$\cdot $ &$e$ &$a$ &$b$ &$c$ \\
		\hline
		$e$ &$e$&$a$&$b$&$c$\\
		$a$ &$a$&$e$&$c$&$b$\\
		$b$ &$b$&$c$&$e$&$a$\\
		$c$ &$c$&$b$&$a$&$e$
	\end{tabular}
	\end{center}
	A great example of an equivalent group up to isomorphism is the subgroup $\left\{ R_0,R_{180},S,R_{180}S \right\} $ of the dihedral group $D_n$ for an even integer $n$.

	The group is clearly not cyclic. It has 4 elements, none of which generate the group. Since $a^2=b^2=c^2=e$, we have
	\[
		\left<a \right> = \left\{ e,a \right\} 
	,\]
	\[
		\left<b \right> =\left\{ e,b \right\} 
	,\]
	\[
		\left<c \right> =\left\{ e,c \right\} 
	.\]
	Therefore, the set is not cyclic. Additionally, these sets and the set $\left\{ e \right\} $ are the only possible cyclic subgroups of $G$, which is obvious since they are the sets generated by each element of $G$. It remains to be shown no other subgroups of $G$ exist.

	Let $H\subsetneq G$ have more than 2 elements. Since any set must have the identity element in it, we have already seen all the subgroups of $G$ with 2 or less elements. They are exactly the cyclic subgroups of $G$. We must only consider sets with more than 2 but less than 4 elements. In other words, we must consider sets with only 3 elements. All of these sets, in order to be groups, must have the identity element, so they must have exactly 2 non-identity elements $a $, $b$, or $c$ in them. Since $ab=ba=c$, $bc=cb=a$, and $ca=ac=b$, none of these sets would be closed under group multiplication. Thus, there exist no proper subgroups of $G$ other than the cyclic subgroups.
\end{solution}
\section*{Problem 3}
\begin{solution}
	
\end{solution}
\section*{Problem 4}
\begin{solution}
	Let $G$ be a group with identity $e$. Let $a,b\in G$ such that $\left| a \right| =n$, $\left| b \right| =m$, and $\operatorname{gcd}(n,m) =1$.

	Let there exist some $g\in \left<a \right>$ such that $g\in \left<b \right>$. Since $g\in \left<a \right>$, there must exist some $k\in\mathbb{Z}$ such that $g=a^k$. Since $\left<b \right>$ is a group and thus closed under multiplication, any power of $a^k$ must be an element of $\left<b \right>$. Thus, we have
	\[
		\left<a^k \right>\subseteq \left<b \right>
	.\]
	By the fundamental theorem of cyclic subgroups, we know that $\left| \left<a^k \right> \right| $ will divide $m$. Additionally, by the same logic $\left<a^k \right>\subseteq \left<a \right>$, so $\left| \left<a^k \right> \right| $ must also divide $n$. Since $m$ and $n$ are relatively prime, the only number that divides both $m$ and $n$ is $1$, and thus $\left| \left<a^k \right> \right| =1$. It follows from this that
	\[
		\left( a^k \right) ^1=e
	.\]
	Thus,
	\[
		a^k=e
	.\]
	Therefore, $g\in \left<a \right>$ and $g\in \left<b \right>$ implies $g=e$. This should suffice to show $\left<a \right>\cap \left<b \right> = \left\{ e \right\} $.
\end{solution}
\section*{Problem 5}
\begin{solution}
	Let $a\in G$. The set $C(a)$ is the set of all $g\in G$ such that $ga=ag$. We need only show $\left<a \right>\subseteq C(a)$. Since $a^k\in \left<a \right>$ for any $k\in\mathbb{Z}$, we have
	\[
		a\left( a^k \right) =a\left( a^{k-1}a \right) =\left( aa^{k-1} \right) a=a^{k}a
	.\]
	Therefore, for any $a^k\in \left<a \right>$, $a^ka=aa^k$. Thus, $\left<a \right>\subseteq C(a)$.
\end{solution}
\end{document}
