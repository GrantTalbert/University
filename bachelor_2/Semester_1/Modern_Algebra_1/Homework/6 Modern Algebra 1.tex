\documentclass[11pt, letterpaper]{report}
%basic packages
\usepackage[utf8]{inputenc}
\usepackage[T1]{fontenc}
\usepackage{graphicx}
\usepackage[margin=0.75in]{geometry}
\usepackage[usenames,dvipsnames]{xcolor}

%math
\usepackage{amsmath, amsthm, amsfonts, amssymb, mathtools}
\usepackage{mathrsfs}
\usepackage{cancel}
\usepackage{siunitx} %phyjsicsssss
\usepackage{bbm} %mathbb for numbers
\usepackage[all]{xy} % https://texdoc.org/serve/xyguide.pdf/0
\makeatletter
\renewcommand*\env@matrix[1][c]{\hskip -\arraycolsep
  \let\@ifnextchar\new@ifnextchar
  \array{*\c@MaxMatrixCols #1}}
\makeatother %matrix realignment

%misc
\usepackage{float}
\usepackage[hyphens]{url}
%\definecolor{page}{HTML}{242526}
%\pagecolor{page}
\usepackage{booktabs} %the \toprule and \bottomrule thick lines on tables

\usepackage{hyperref}
\definecolor{aqua}{HTML}{00C5FF}
\hypersetup{
    colorlinks,
    linkcolor={aqua},
    urlcolor={aqua},
    citecolor={red}
}

%my commands
\DeclarePairedDelimiter\bra{\langle}{\rvert} %Bra
\DeclarePairedDelimiter\ket{\lvert}{\rangle} %Ket
\DeclarePairedDelimiterX\braket[2]{\langle}{\rangle}{#1\,\delimsize\vert\,\mathopen{}#2} %Bra-ket
\newcommand{\pvec}[1]{\vec{#1}\mkern2mu\vphantom{#1}} % from https://tex.stackexchange.com/questions/120029/how-to-typeset-a-primed-vector
\newcommand{\hati}{\boldsymbol{\hat{\textbf{\i}}}}
\newcommand{\hatj}{\boldsymbol{\hat{\textbf{\j}}}}
\newcommand{\hatk}{\boldsymbol{\hat{\textbf{k}}}}
\newcommand{\R}{\mathbb{R}}
\DeclareMathOperator{\diag}{diag}

%theorems
\usepackage{thmtools}
\usepackage{tikz}
\usepackage{tikz-cd}
\usepackage[framemethod=TikZ]{mdframed}
\mdfsetup{skipabove=1em,skipbelow=0em, innertopmargin=5pt, innerbottommargin=6pt}
\theoremstyle{definition} %because obviously
% THEOREM STYLES

\definecolor{boxColor}{HTML}{3D3D3D}

% Definitions
\newcounter{def}[chapter]\setcounter{def}{0}
\renewcommand{\thedef}{\arabic{chapter}.\arabic{def}}
\newenvironment{definition}[1][]{
  \refstepcounter{def}
  \ifstrempty{#1}{
    \mdfsetup{
      frametitle = {
        \tikz[baseline=(current bounding box.east),outer sep=0pt]
        \node[anchor=east,rectangle,font=\bfseries,fill=white]{\strut Definition~\thedef};}}}
  {\mdfsetup{
      frametitle={
        \tikz[baseline=(current bounding box.east),outer sep=0pt]
        \node[anchor=east,rectangle,font=\bfseries,fill=white]{\strut Definition~\thedef :~#1};}}}
  \mdfsetup{innertopmargin=-5pt,linewidth=1pt,topline=true,nobreak,frametitleaboveskip=\dimexpr-12pt\relax\strutbox}
\begin{mdframed}[]}{\end{mdframed}}

% Theorems
\newcounter{thm}[chapter]\setcounter{thm}{0}
\renewcommand{\thethm}{\arabic{chapter}.\arabic{thm}}
\newenvironment{theorem}[1][]{
  \refstepcounter{thm}
  \ifstrempty{#1}{
    \mdfsetup{
      frametitle = {
        \tikz[baseline=(current bounding box.east),outer sep=0pt]
        \node[anchor=east,rectangle,font=\bfseries,fill=white,draw,thick]{\strut Theorem~\thethm};}}}
  {\mdfsetup{
      frametitle={
        \tikz[baseline=(current bounding box.east),outer sep=0pt]
        \node[anchor=east,rectangle,font=\bfseries,fill=white,draw,thick]{\strut Theorem~\thethm~(#1)};}}}
  \mdfsetup{innertopmargin=2pt,linewidth=1pt,topline=true,nobreak,frametitleaboveskip=\dimexpr-13pt\relax\strutbox}
\begin{mdframed}[]}{\end{mdframed}}

% Lemmas
\newcounter{lma}[chapter]\setcounter{lma}{0}
\renewcommand{\thelma}{\arabic{chapter}.\arabic{lma}}
\newenvironment{lemma}[1][]{
  \refstepcounter{lma}
  \ifstrempty{#1}{
    \mdfsetup{
      frametitle = {
        \tikz[baseline=(current bounding box.east),outer sep=0pt]
        \node[anchor=east,rectangle,font=\bfseries,fill=white,draw,thick]{\strut Lemma~\thelma};}}}
  {\mdfsetup{
      frametitle={
        \tikz[baseline=(current bounding box.east),outer sep=0pt]
        \node[anchor=east,rectangle,font=\bfseries,fill=white,draw,thick]{\strut Lemma~\thelma~(#1)};}}}
  \mdfsetup{innertopmargin=2pt,linewidth=1pt,topline=true,nobreak,frametitleaboveskip=\dimexpr-13pt\relax\strutbox}
\begin{mdframed}[]}{\end{mdframed}}


% Corollaries
\newcounter{corr}[chapter]\setcounter{corr}{0}
\renewcommand{\thecorr}{\arabic{chapter}.\arabic{corr}}
\newenvironment{corollary}[1][]{
  \refstepcounter{corr}
  \ifstrempty{#1}{
    \mdfsetup{
      frametitle = {
        \tikz[baseline=(current bounding box.east),outer sep=0pt]
        \node[anchor=east,rectangle,font=\bfseries,fill=white,draw,thick]{\strut Corollary~\thecorr};}}}
  {\mdfsetup{
      frametitle={
        \tikz[baseline=(current bounding box.east),outer sep=0pt]
        \node[anchor=east,rectangle,font=\bfseries,fill=white,draw,thick]{\strut Corollary~\thecorr~(#1)};}}}
  \mdfsetup{innertopmargin=2pt,linewidth=1pt,topline=true,nobreak,frametitleaboveskip=\dimexpr-13pt\relax\strutbox}
\begin{mdframed}[]}{\end{mdframed}}

% Propositions
\newcounter{prp}[chapter]\setcounter{prp}{0}
\renewcommand{\theprp}{\arabic{chapter}.\arabic{prp}}
\newenvironment{proposition}[1][]{
  \refstepcounter{prp}
  \ifstrempty{#1}{
    \mdfsetup{
      frametitle = {
        \tikz[baseline=(current bounding box.east),outer sep=0pt]
        \node[anchor=east,rectangle,font=\bfseries,fill=white,draw,thick]{\strut Proposition~\theprp};}}}
  {\mdfsetup{
      frametitle={
        \tikz[baseline=(current bounding box.east),outer sep=0pt]
        \node[anchor=east,rectangle,font=\bfseries,fill=white,draw,thick]{\strut Proposition~\theprp~(#1)};}}}
  \mdfsetup{innertopmargin=2pt,linewidth=1pt,topline=true,nobreak,frametitleaboveskip=\dimexpr-13pt\relax\strutbox}
\begin{mdframed}[]}{\end{mdframed}}

% Remarks
\newenvironment{remark}[1][]{
  \mdfsetup{
      frametitle={
        \tikz[baseline=(current bounding box.east),outer sep=0pt]
        \node[anchor=east,rectangle,font=\bfseries,fill=white]{\strut Remark};}}
  \mdfsetup{innertopmargin=-7pt,linewidth=1pt,topline=true,bottomline=true,leftline=false,rightline=false,nobreak,frametitleaboveskip=\dimexpr-12pt\relax\strutbox}
\begin{mdframed}[]}{\end{mdframed}}

% Examples
\newenvironment{example}[1][]{
  \mdfsetup{
      frametitle={
        \tikz[baseline=(current bounding box.east),outer sep=0pt]
        \node[anchor=east,rectangle,font=\bfseries,fill=white,draw,thick]{\strut Example};}}
  \mdfsetup{innertopmargin=0pt,linewidth=1pt,topline=true,bottomline=true,leftline=false,rightline=false,nobreak,frametitleaboveskip=\dimexpr-13pt\relax\strutbox}
\begin{mdframed}[]}{\end{mdframed}}

% As Previously Seen
\newenvironment{prev}[1][]{
  \mdfsetup{
      frametitle={
        \tikz[baseline=(current bounding box.east),outer sep=0pt]
        \node[anchor=east,rectangle,font=\bfseries,fill=white]{\strut As Previously Seen};}}
  \mdfsetup{innertopmargin=-7pt,linewidth=1pt,topline=true,bottomline=true,leftline=false,rightline=false,nobreak,frametitleaboveskip=\dimexpr-12pt\relax\strutbox}
\begin{mdframed}[]}{\end{mdframed}}

% Exercises
\newcounter{exe}[chapter]\setcounter{exe}{0}
\renewcommand{\theexe}{\arabic{chapter}.\arabic{exe}}
\newenvironment{exercise}[1][]{
  \refstepcounter{exe}
  \mdfsetup{
      frametitle={
        \tikz[baseline=(current bounding box.east),outer sep=0pt]
        \node[anchor=east,rectangle,font=\bfseries,fill=white,draw,thick]{\strut Exercise~\theexe};}}
  \mdfsetup{roundcorner=10pt,innertopmargin=0pt,linewidth=1pt,topline=true,nobreak,frametitleaboveskip=\dimexpr-13pt\relax\strutbox}
\begin{mdframed}[]}{\end{mdframed}}

%theorem styles
\declaretheoremstyle[headfont=\bfseries, bodyfont=\normalfont, mdframed={linewidth=1pt,   bottomline=false, topline=false, rightline=false}, qed=\(\blacksquare\)]{proofline}
\declaretheoremstyle[headfont=\bfseries, bodyfont=\normalfont, mdframed={linewidth=1pt,   bottomline=false, topline=false, rightline=false}, qed=\qedsymbol]{egline}

\declaretheorem[numbered=no, name=Notation]{notation}

%solution environment
\declaretheorem[numbered=no, style=egline, name=Solution]{setsolution}
\newenvironment{solution}[1][]{\vspace{-10pt}\begin{setsolution}}{\end{setsolution}}
%proof environment
\declaretheorem[numbered=no, style=proofline, name=Proof]{replacementproof}
\renewenvironment{proof}[1][\proofname]{\begin{replacementproof}}{\end{replacementproof}}

% Side Indented Theorems - https://tex.stackexchange.com/questions/429339/shifting-newtheorem
\newtheoremstyle{side}{}{}{\advance\leftskip3cm\relax\itshape\normalfont}{-4pt}
{\bfseries}{}{0pt}{
\makebox[0pt][r]{
  \smash{\parbox[t]{2.5cm}{\raggedright\thmname{#1}.
  \thmnote{\newline(#3)}}}
  \hspace{10.1pt}}}

\theoremstyle{side}
\newtheorem{note}{Note}
\newtheorem{intuition}{Intuition}
\newtheorem{claim}{Claim}
\theoremstyle{definition}

%pulls lecture files
\newcommand{\lec}[2]{%
	\foreach \c in {#1,...,#2}{%
		\IfFileExists{Lectures/lec_\c.tex} {%
			\input{Lectures/lec_\c.tex}%
		}{}%
	}%
}
%to use, in the same file directory as your header.tex and master.tex files, create a folder titled "Lectures" and put your
%lectures into their, named lec_1.tex, lec_2.tex, and so on. In the master.tex file, write \lec{a}{b} where a is the lowest
%number you want to call, and b is the highest.

%fancy headers
\usepackage{fancyhdr}
\pagestyle{fancy}
\fancyhead{}\fancyfoot{}
\fancyfoot[R]{\thepage}
\fancyfoot[C]{\leftmark}


%figures, taken from (https://castel.dev/post/lecture-notes-2)
\usepackage{import}
\usepackage{xifthen}
\usepackage{pdfpages}
\usepackage{transparent}
\newcommand{\incfig}[1]{%
    \def\svgwidth{\columnwidth}
    \import{./figures/}{#1.pdf_tex}
}

%lectures, taken from (https://castel.dev/post/lecture-notes-3)
\makeatother
\def\@lecture{}%
\newcommand{\lecture}[3]{%
	\ifthenelse{\isempty{#3}}{%

		\def\@lecture{Lecture #1}%
	}{%
		\def\@lecture{Lecture #1: #3}
	}
	\subsection*{\@lecture}
	\hfill{\small\textsf{#2}}\par
}
\makeatletter

\author{Grant Talbert}

\usepackage{titlepageBU}
\title{Modern Algebra 1}
\date{09/19/24}
\courseID{MA 541}
\professor{Dr. Duque-Rosero}
\courseSection{A1}
\renewcommand{\mod}[1]{\;(\text{mod}\;#1)}
\renewenvironment{solution}[1][]{\begin{setsolution}}{\end{setsolution}}
\begin{document}
 \makeproblem
\section*{Problem 1}
\begin{solution}
	(a) $\operatorname{gcd}(12,40) =4$. This should be obvious, but I will give some further reasoning because why not. The only divisors of $12$ greater than $4$ are 6 and 12, and neither of these divide $40$.

	Since $p,q$ are prime, they have no common divisors. Thus, any divisor of any product of $p$ and $q$ must also be a product of $p$ and $q$. It should follow that the greatest common divisor is the product of the highest powers of $p$ and $q$ that are in both primes, since any higher power and it would no longer divide one of the numbers. This gives us $\operatorname{gcd}(p^2q^2,pq^3) =pq^2$.

	(b) Let $a,b\in\mathbb{Z}$ have $\operatorname{gcd}(a,b) =d$. We know there exist $q_1,r_1\in\mathbb{Z}$ such that
	\[
		a=q_1b+r_1
	\]
	and  $0\leq r_1<\left| b \right| $. For $r_1=0$, we have $b=d$, since $a=q_1b$ implies $b$ would divide $a$, and trivially $b$ divides $b$. In this case, we are done since
	\[
		d=b=0a+1b
	,\]
	and $0,1\in\mathbb{Z}$. For $r_1\neq 0$, we have
	\[
		b = q_2r_1+r_2
	,\]
	again with $q_2,r_2\in\mathbb{Z}$ and $0\leq r_2<r_1$. We drop the absolute value signs here, since $r_1$ is positive, and thus $r_2$ is also positive. Since $r_1<\left| b \right| $, and $r_2<r_1$, and $r_1,r_2,b\in\mathbb{Z}$, we have the maximum value of $r_2 = r_1 - 1 = b - 2$. We can repeat this process $k$ times, until $r_k = 0$. At this point, we have
	\[
		r_{k-2}=q_kr_{k-1}
	.\]
	Thus, $r_{k-1}$ divides $b$. Now, we have to rearrange all of these equations. Since $a=q_1b+r_1$, we have
	\[
		r_1=a-q_1b
	.\]
	We also know that $b=q_2r_1 + r_2$, so we can plug in for $r_1$.
	\[
		b = q_2 \left( a-q_1b \right) +r_2 = q_2a-q_1q_2b+r_2
	.\]
	Rearranging, we have
	\[
		r_2 = b-q_2a+q_1q_2b = (1+q_1q_2)b-q_2a
	.\]
	In our next equation, we have
	\[
		r_1 = q_3r_2 + r_3
	.\]
	We can plug in for $r_1$ and $r_2$, and obtain
	\[
		a-q_1b = q_3\left( \left( 1+q_1q_2 \right) b-q_2a \right) +r_3
	.\]
	It follows that
	\[
		(1-q_2q_3)a + \left( q_3 + q_1q_2q_3-q_1 \right) b= r_3
	.\]
	We notice that for any $r_i$, we can rearrange statements to show that $r_i = pa+qb$ for integers $p,q$. We continue this process for all $k$ statements, and collect this mess of coefficients into integers $m,n$. Eventually, we reach
	\[
		r_{k-2}=(ma+nb)
	,\]
	since we had $r_k=0$, and $q_kr_{k-1}=ma+nb$. We can now backtrack. Since $r_{k-3}=r_{k-2}q_{k-1}+r_{k-1}$, we can plug in our known values, and without loss of generality, absorb the coefficients into our integers $m,n$. Eventually, we have
\[
	b = q_2r_1 + r_2 = ma+nb
.\]
It follows that, again without loss of generality since we are absorbing into our coefficients,
\[
	ma=-nb
.\]
	It follows that $r_{k-1}=d$ or something i dont know i got lost 4 hours ago.

	(c) Let $\operatorname{gcd}(a,b) =1$. By theorem A, it follows that there exist $m,n\in \mathbb{Z}$ such that $ma+nb=1$. Now we prove the converse. Let there exist $m,n\in\mathbb{Z}$ such that $ma+nb=1$. Since $1$ is the smallest positive integer, it must be the smallest positive integer that can be written as $ma+nb$. It follows from theorem A that $\operatorname{gcd}(a,b) =1$. Therefore, $a$ and $b$ are relatively prime if and only if there exist integers $m,n\in\mathbb{Z}$ such that $ma+nb=1$.
\end{solution}
\section*{Problem 2}
\begin{solution}
	(a) Let $a\in\mathbb{Z}^+$. Let $\operatorname{gcd}(a,n) =1$. By part (c) of problem 1, it follows that there exists integers $x,k\in\mathbb{Z}$ such that $ax+kn=1$. It should be obvious that $ax+kn\equiv ax\mod{n}$. Thus, it follows that $ax\equiv 1\mod{n}$. Now we prove the converse. Let $ax\equiv 1\mod{n}$. It follows that there exists some $k$ such that $ax = 1 + kn$. Thus, $ax - kn = 1$. Without loss of generality, and to avoid having to use negative signs, redefine $k$ such that $ax + kn = 1$. By part (c) of problem 1, since there exist $x,k\in\mathbb{Z}$ such that $ax+kn=1$, it follows that $\operatorname{gcd}(a,n) =1$.

(b) Clearly, $U(n)$ is not a subgroup of $\mathbb{Z}_n$, since it is defined under \emph{multiplication} modulo $n$, not \emph{addition} modulo $n$. It remains to be shown $U(n)$ is actually a group.

First, we show closure under multiplication modulo $n$. Let $a,b\in U(n)$. It follows that $\operatorname{gcd}(a,n) =\operatorname{gcd}(b,n) =1$. From part (c) of problem 1, we have
\[
	r_1a+k_1n=1
\]
and
\[
	r_2b+k_2n=1
\]
for $r_1,r_2,k_1,k_2\in\mathbb{Z}$. We wish to show $ab$ is relatively prime to $n$. We have
\begin{align*}
	1\cdot 1&=\left( r_1a+k_1n \right) \left( r_2b+k_2n \right) \\
		&=\left( r_1r_2 \right) ab+r_1ak_2n+r_2bk_1n+k_1k_2n^2\\
		&=\left( r_1r_2 \right) ab+\left( r_1ak_2+r_2bk_1+k_1k_2n \right) n
.\end{align*}
It follows that there exist integers $\alpha \coloneqq r_1r_2$ and $\beta \coloneqq r_1ak_2+r_2bk_1+k_1k_2n$ such that $\alpha ab+\beta n=1$. Therefore, by part (c) of problem 1, $ab$ is relatively prime to $n$ if $a$ and $b$ are relatively prime to $n$. It remains to be shown that $ab\mod{n}$ is relatively prime to $n$.

By definition, there exists some integers $s,\lambda $ such that $ab\equiv s+n\lambda \equiv s \mod{n}$, where we take $s$ to be the minimum possible of the set $\left\{ 0,\ldots,n-1 \right\} $. This is an equivalent statement to theorem A, of which part (c) of problem 1 is a special case. Let $\lambda =r_1aak_2+r_2bk_1+k_1k_2n$. It follows that $ab\equiv 1\mod{n}$. Thus, $ab\mod{n}$ is relatively prime to $n$, and we have closure under multiplication modulo $n$.

Associativity follows trivially from known properties of the integers. Additionally, $\operatorname{gcd}(1,n) =1$ for any $n$, so we have the identity element $1\in U(n)$. We must show now that inverses exist for any $a\in U(n)$. By part (a) of problem 2, we have for any $a$ relatively prime to $n$, and thus any $a\in U(n)$, there exists some $b$ such that $ab\equiv 1\mod{n}$. We must show such a $b$ is relatively prime to $n $. Since $a$ is relatively prime to $n$, and by the logic in the proof of part (a), we have
\[
	ab+kn=1
.\]
From this, we know there exist integers $a,k$ such that $ab+kn=1$, and thus $\operatorname{gcd}(b,n) =1$. Therefore, for any $a\in U(n)$, there exists some $b\in U(n)$ such that $ab=ba=1$.

(c) By the definition of the function, $\left| U(n) \right| =\phi (n)$, where $\phi $ is the Euler totient function.
\end{solution}
\section*{Problem 3}
For this problem, for the set $X$, the notation $\left| X \right| $ indicates \emph{the number of elements of} $X$. This is important to clarify, because although $X$ is probably a subgroup of $G$, I don't really feel like checking, and we want to avoid ambiguous notation.
\begin{solution}
	For convenience of notation, let $X$ be the set of all $x\in G$ such that $x^n=e$.

	Let $x\in X$. Since $x^n=e$, we have $xx^{n-1}=e$. It follows that $x ^{-1}=x^{n-1}$. We also have
	\[
		\left( x^{n-1} \right) ^n = x^{n^2-n}=x^{n^2}x^{-n}=\left( x^n \right) ^n \left( x ^{n} \right) ^{-1} = e^n e^{-1}=e
	.\]
	Thus, if $x\in X$, then $x ^{-1}\in X$.

	Now, note that $e^n = e$, and thus $e\in X$. If we can show that for any $x\in X\setminus\left\{ e \right\} $ that $x\neq x ^{-1}$, then we will have finished the proof, because since inverses are unique, we would then have $X$ consisting of \emph{pairs} of elements $x,x ^{-1}$, except for $e$, which would have no other element to pair with. Since each pair has two elements, we have $\left| X \right| =2n+1$ for $n$ pairs in $X$, since there are two elements per pair plus an extra element from $e$.

	Showing that for $x\neq e$, $x \neq x ^{-1}$ is actually rather easy. Since $n$ is odd, $n-1$ is even. Since $x=x^1$, and 1 is odd, we have $x^{n-1} \neq x$, since the odd number 1 cannot equal the even number $n-1$. Since $x^{n-1}= x ^{-1}$, we have $x \neq x ^{-1}$.
\end{solution}
\section*{Problem 4}
\begin{solution}
	Obviously, this subgroup will be $H=\left\{R_0,R_{180},S,R_{180}S\right\}$. To show this is true, we must show the following three things:
	\begin{itemize}
		\item $R_{180}\in D_{n}$ for $n$ even. Since $R_0,S\in D_n$ for any $n$, the only ambiguity is in $R_{180}$ and $R_{180}S$, both of which can be shown by proving the former. This implies $H\subseteq D_n$.
		\item $H$ is closed under function composition.
		\item $H$ is a group.
	\end{itemize}
	We begin by proving the first one. A full rotation of any polygon is a rotation by $360$ degress. Define $R$ to be the smallest rotation that is an element of $D_n$. We know that $R$ is a rotation by $360/n$ degrees, so $R^n$ is a rotation by 360 degrees. Since $180$ is half of $360$, we know $R^{n/2}$ would be a rotation by $180$ degrees, if $\frac{n}{2}\in\mathbb{Z}$. Since $n$ is even, it is divisible by $2$, and thus $\frac{n}{2}\in\mathbb{Z}$, and thus $R^{\frac{n}{2}}=R_{180}\in D_n$. Since $S\in D_n$, and $D_n$ is a group and thus closed, $R_{180}S\in D_n$.

	For the next two points, it suffices to present a Cayley Table.
	\begin{center}
		\begin{tabular}{c|cccc}
			$\circ $ &$R_{0}$ &$R_{180}$ &$S$ &$R_{180}S$ \\
			\hline
			$R_0$ &$R_{0}$&$R_{180}$&$S$&$R_{180}S$\\
			$R_{180}$ &$R_{180}$&$R_0$&$R_{180}S$&$S$\\
			$S$ &$S$&$R_{180}S$&$R_0$&$R_{180}$\\
			$R_{180}S$ &$R_{180}S$&$S$&$R_{180}$&$R_0$
		\end{tabular}
	\end{center}
	This table assumes two facts that are worthy of proof:
	\begin{itemize}
		\item $R_{180}S=SR_{180}$.
		\item $R_{180}SR_{180}S=R_0$.
	\end{itemize}
	The second is a corollary of the first. Since $R^{i}S=SR^{-i}$, we have
	\[
		R_{180}S=R^{\frac{n}{2}}S=SR^{-\frac{n}{2}}
	.\]
	This is a rotation in the opposite direction by 180 degrees. However, a rotation in the opposite direction by 180 degrees is the same as a rotation in the original direction by 180 degrees, so
	\[
		R^{-\frac{n}{2}}=R^{\frac{n}{2}}
	.\]
	Thus,
	\[
		R_{180}S=SR_{180}
	.\]
	By this fact, we have
	\[
		(R_{180}S)(R_{180}S)=\left( R_{180}S \right) \left( SR_{180} \right) =R_{180}(SS)R_{180}=R_{180}R_{180}=R_0
	.\]
	By the same logic, we also have $SR_{180}S=SSR_{180}=R_{180}$.
	The Cayley table thus clearly shows inverses exist for each element, an identity $R_0$ exists, the set is closed, and associativity will follow from the fact that $D_n$ is a group. Thus, $H$ is a group. Since $H\subseteq D_n$, $H$ is a subgroup of $D_n$ for any even $n$. Since $H$ is finite with $4$ elements, the order $\left| H \right| =4$.
\end{solution}
\section*{Problem 5}
This solution makes extensive use of the theorem that for any group $G$, $H\subseteq G$ is a group if and only if $H\neq \varnothing$, and for any $a,b\in G$, $ba^{-1}\in G$.
\begin{solution}
	(a) Let $G$ be a group with subgroups $H,K$. Since $H,K\subseteq G$, we have $H\cap K\subseteq G$. It follows that for the identity element $e$ of $G$, we have $e\in H$ and $e\in K$, since any group requires an identity element, and thus $e\in H\cap K$. Thus, $H\cap K\neq \varnothing$. It remains to be shown that for $h,g\in H\cap K$, $gh^{-1}\in H\cap K$.

	Let $h,g\in H\cap K$. It follows immedately that $h,g\in H$ and $h,g\in K$. Since a set $H\subseteq G$ is a subgroup \emph{if and only if} $h,g\in H$ implies $gh^{-1}\in H$, we thus know $gh^{-1}\in H$ and $gh\in K$ by the reverse direction of this theorem. Since $gh^{-1}\in H$ and $gh^{-1}\in K$, we have $gh^{-1}\in H\cap K$, and thus $H\cap K$ is a subgroup of $G$.

	(b) Since $H \not\subset K$, there must exist at least one $h\in H$ such that $h\notin K$. Conversely, since $K\not\subset H$, there must exist at least one $k\in K$ such that $k\notin H$. We also know tht $k,h\in K\cup H$. Thus, $K\cup H \neq K,H$. By this logic, let $k\in K$ and $h\in H$ such that $k\notin H$ and $h\notin K$. Since $H$ is a subgroup of $G$, and thus a group, any element of $H$ must have an inverse in $H$. Therefore, $h\in H$ implies that $h^{-1}\in H$. By this same logic, $h^{-1}\notin K$, since $h^{-1}\in K$ would imply $h\in K$, which is a contradiction. The same logic implies $k^{-1}\in K$ and $k^{-1}\notin H$. We thus have $h,h^{-1},k,k^{-1}\in H\cup K$.

	Now suppose for purpose of contradiction that $H\cup K$ is a group. It follows that since $h,k\in H\cup K$, we have $kh^{-1}\in H\cup K$. As a consequence, at least one of the following two statements are true:
	\begin{itemize}
		\item $kh^{-1}\in H$.
		\item $kh^{-1}\in K$.
	\end{itemize}
	Suppose for now that $kh^{-1}\in H$. Since $kh^{-1},h\in H$, and since $H$ is a group and thus closed under group multiplication, we have $kh^{-1}h\in H$. It follows that
	\[
		kh^{-1}h=ke=k\in H
	.\]
	This is a contradiction, implying that $kh^{-1}\notin H$.

	Now suppose $kh^{-1}\in K$. Again, since $kh^{-1},k^{-1}\in K$, and since $K$ is a group and thus closed under group multiplication, we have $k^{-1}kh^{-1}\in K$. It follows that
	\[
		k^{-1}kh^{-1}=eh^{-1}=h^{-1}\in K
	.\]
	This is again a contradiction, implying that $kh^{-1}\notin K$.

	Since  $kh^{-1}\notin K$ and $kh^{-1}\notin H$, we have $kh^{-1}\notin H\cup K$, which is a contradiction since we assumed $H\cup K$ was a group. Thus, $H\cup K$ is not a group, and thus is not a subgroup of $G$.
\end{solution}
\end{document}
