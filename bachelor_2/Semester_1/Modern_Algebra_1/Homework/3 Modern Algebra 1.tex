\documentclass[11pt, letterpaper]{report}
%basic packages
\usepackage[utf8]{inputenc}
\usepackage[T1]{fontenc}
\usepackage{graphicx}
\usepackage[margin=0.75in]{geometry}
\usepackage[usenames,dvipsnames]{xcolor}

%math
\usepackage{amsmath, amsthm, amsfonts, amssymb, mathtools}
\usepackage{mathrsfs}
\usepackage{cancel}
\usepackage{siunitx} %phyjsicsssss
\usepackage{bbm} %mathbb for numbers
\usepackage[all]{xy} % https://texdoc.org/serve/xyguide.pdf/0
\makeatletter
\renewcommand*\env@matrix[1][c]{\hskip -\arraycolsep
  \let\@ifnextchar\new@ifnextchar
  \array{*\c@MaxMatrixCols #1}}
\makeatother %matrix realignment

%misc
\usepackage{float}
\usepackage[hyphens]{url}
%\definecolor{page}{HTML}{242526}
%\pagecolor{page}
\usepackage{booktabs} %the \toprule and \bottomrule thick lines on tables

\usepackage{hyperref}
\definecolor{aqua}{HTML}{00C5FF}
\hypersetup{
    colorlinks,
    linkcolor={aqua},
    urlcolor={aqua},
    citecolor={red}
}

%my commands
\DeclarePairedDelimiter\bra{\langle}{\rvert} %Bra
\DeclarePairedDelimiter\ket{\lvert}{\rangle} %Ket
\DeclarePairedDelimiterX\braket[2]{\langle}{\rangle}{#1\,\delimsize\vert\,\mathopen{}#2} %Bra-ket
\newcommand{\pvec}[1]{\vec{#1}\mkern2mu\vphantom{#1}} % from https://tex.stackexchange.com/questions/120029/how-to-typeset-a-primed-vector
\newcommand{\hati}{\boldsymbol{\hat{\textbf{\i}}}}
\newcommand{\hatj}{\boldsymbol{\hat{\textbf{\j}}}}
\newcommand{\hatk}{\boldsymbol{\hat{\textbf{k}}}}
\newcommand{\R}{\mathbb{R}}
\DeclareMathOperator{\diag}{diag}

%theorems
\usepackage{thmtools}
\usepackage{tikz}
\usepackage{tikz-cd}
\usepackage[framemethod=TikZ]{mdframed}
\mdfsetup{skipabove=1em,skipbelow=0em, innertopmargin=5pt, innerbottommargin=6pt}
\theoremstyle{definition} %because obviously
% THEOREM STYLES

\definecolor{boxColor}{HTML}{3D3D3D}

% Definitions
\newcounter{def}[chapter]\setcounter{def}{0}
\renewcommand{\thedef}{\arabic{chapter}.\arabic{def}}
\newenvironment{definition}[1][]{
  \refstepcounter{def}
  \ifstrempty{#1}{
    \mdfsetup{
      frametitle = {
        \tikz[baseline=(current bounding box.east),outer sep=0pt]
        \node[anchor=east,rectangle,font=\bfseries,fill=white]{\strut Definition~\thedef};}}}
  {\mdfsetup{
      frametitle={
        \tikz[baseline=(current bounding box.east),outer sep=0pt]
        \node[anchor=east,rectangle,font=\bfseries,fill=white]{\strut Definition~\thedef :~#1};}}}
  \mdfsetup{innertopmargin=-5pt,linewidth=1pt,topline=true,nobreak,frametitleaboveskip=\dimexpr-12pt\relax\strutbox}
\begin{mdframed}[]}{\end{mdframed}}

% Theorems
\newcounter{thm}[chapter]\setcounter{thm}{0}
\renewcommand{\thethm}{\arabic{chapter}.\arabic{thm}}
\newenvironment{theorem}[1][]{
  \refstepcounter{thm}
  \ifstrempty{#1}{
    \mdfsetup{
      frametitle = {
        \tikz[baseline=(current bounding box.east),outer sep=0pt]
        \node[anchor=east,rectangle,font=\bfseries,fill=white,draw,thick]{\strut Theorem~\thethm};}}}
  {\mdfsetup{
      frametitle={
        \tikz[baseline=(current bounding box.east),outer sep=0pt]
        \node[anchor=east,rectangle,font=\bfseries,fill=white,draw,thick]{\strut Theorem~\thethm~(#1)};}}}
  \mdfsetup{innertopmargin=2pt,linewidth=1pt,topline=true,nobreak,frametitleaboveskip=\dimexpr-13pt\relax\strutbox}
\begin{mdframed}[]}{\end{mdframed}}

% Lemmas
\newcounter{lma}[chapter]\setcounter{lma}{0}
\renewcommand{\thelma}{\arabic{chapter}.\arabic{lma}}
\newenvironment{lemma}[1][]{
  \refstepcounter{lma}
  \ifstrempty{#1}{
    \mdfsetup{
      frametitle = {
        \tikz[baseline=(current bounding box.east),outer sep=0pt]
        \node[anchor=east,rectangle,font=\bfseries,fill=white,draw,thick]{\strut Lemma~\thelma};}}}
  {\mdfsetup{
      frametitle={
        \tikz[baseline=(current bounding box.east),outer sep=0pt]
        \node[anchor=east,rectangle,font=\bfseries,fill=white,draw,thick]{\strut Lemma~\thelma~(#1)};}}}
  \mdfsetup{innertopmargin=2pt,linewidth=1pt,topline=true,nobreak,frametitleaboveskip=\dimexpr-13pt\relax\strutbox}
\begin{mdframed}[]}{\end{mdframed}}


% Corollaries
\newcounter{corr}[chapter]\setcounter{corr}{0}
\renewcommand{\thecorr}{\arabic{chapter}.\arabic{corr}}
\newenvironment{corollary}[1][]{
  \refstepcounter{corr}
  \ifstrempty{#1}{
    \mdfsetup{
      frametitle = {
        \tikz[baseline=(current bounding box.east),outer sep=0pt]
        \node[anchor=east,rectangle,font=\bfseries,fill=white,draw,thick]{\strut Corollary~\thecorr};}}}
  {\mdfsetup{
      frametitle={
        \tikz[baseline=(current bounding box.east),outer sep=0pt]
        \node[anchor=east,rectangle,font=\bfseries,fill=white,draw,thick]{\strut Corollary~\thecorr~(#1)};}}}
  \mdfsetup{innertopmargin=2pt,linewidth=1pt,topline=true,nobreak,frametitleaboveskip=\dimexpr-13pt\relax\strutbox}
\begin{mdframed}[]}{\end{mdframed}}

% Propositions
\newcounter{prp}[chapter]\setcounter{prp}{0}
\renewcommand{\theprp}{\arabic{chapter}.\arabic{prp}}
\newenvironment{proposition}[1][]{
  \refstepcounter{prp}
  \ifstrempty{#1}{
    \mdfsetup{
      frametitle = {
        \tikz[baseline=(current bounding box.east),outer sep=0pt]
        \node[anchor=east,rectangle,font=\bfseries,fill=white,draw,thick]{\strut Proposition~\theprp};}}}
  {\mdfsetup{
      frametitle={
        \tikz[baseline=(current bounding box.east),outer sep=0pt]
        \node[anchor=east,rectangle,font=\bfseries,fill=white,draw,thick]{\strut Proposition~\theprp~(#1)};}}}
  \mdfsetup{innertopmargin=2pt,linewidth=1pt,topline=true,nobreak,frametitleaboveskip=\dimexpr-13pt\relax\strutbox}
\begin{mdframed}[]}{\end{mdframed}}

% Remarks
\newenvironment{remark}[1][]{
  \mdfsetup{
      frametitle={
        \tikz[baseline=(current bounding box.east),outer sep=0pt]
        \node[anchor=east,rectangle,font=\bfseries,fill=white]{\strut Remark};}}
  \mdfsetup{innertopmargin=-7pt,linewidth=1pt,topline=true,bottomline=true,leftline=false,rightline=false,nobreak,frametitleaboveskip=\dimexpr-12pt\relax\strutbox}
\begin{mdframed}[]}{\end{mdframed}}

% Examples
\newenvironment{example}[1][]{
  \mdfsetup{
      frametitle={
        \tikz[baseline=(current bounding box.east),outer sep=0pt]
        \node[anchor=east,rectangle,font=\bfseries,fill=white,draw,thick]{\strut Example};}}
  \mdfsetup{innertopmargin=0pt,linewidth=1pt,topline=true,bottomline=true,leftline=false,rightline=false,nobreak,frametitleaboveskip=\dimexpr-13pt\relax\strutbox}
\begin{mdframed}[]}{\end{mdframed}}

% As Previously Seen
\newenvironment{prev}[1][]{
  \mdfsetup{
      frametitle={
        \tikz[baseline=(current bounding box.east),outer sep=0pt]
        \node[anchor=east,rectangle,font=\bfseries,fill=white]{\strut As Previously Seen};}}
  \mdfsetup{innertopmargin=-7pt,linewidth=1pt,topline=true,bottomline=true,leftline=false,rightline=false,nobreak,frametitleaboveskip=\dimexpr-12pt\relax\strutbox}
\begin{mdframed}[]}{\end{mdframed}}

% Exercises
\newcounter{exe}[chapter]\setcounter{exe}{0}
\renewcommand{\theexe}{\arabic{chapter}.\arabic{exe}}
\newenvironment{exercise}[1][]{
  \refstepcounter{exe}
  \mdfsetup{
      frametitle={
        \tikz[baseline=(current bounding box.east),outer sep=0pt]
        \node[anchor=east,rectangle,font=\bfseries,fill=white,draw,thick]{\strut Exercise~\theexe};}}
  \mdfsetup{roundcorner=10pt,innertopmargin=0pt,linewidth=1pt,topline=true,nobreak,frametitleaboveskip=\dimexpr-13pt\relax\strutbox}
\begin{mdframed}[]}{\end{mdframed}}

%theorem styles
\declaretheoremstyle[headfont=\bfseries, bodyfont=\normalfont, mdframed={linewidth=1pt,   bottomline=false, topline=false, rightline=false}, qed=\(\blacksquare\)]{proofline}
\declaretheoremstyle[headfont=\bfseries, bodyfont=\normalfont, mdframed={linewidth=1pt,   bottomline=false, topline=false, rightline=false}, qed=\qedsymbol]{egline}

\declaretheorem[numbered=no, name=Notation]{notation}

%solution environment
\declaretheorem[numbered=no, style=egline, name=Solution]{setsolution}
\newenvironment{solution}[1][]{\vspace{-10pt}\begin{setsolution}}{\end{setsolution}}
%proof environment
\declaretheorem[numbered=no, style=proofline, name=Proof]{replacementproof}
\renewenvironment{proof}[1][\proofname]{\begin{replacementproof}}{\end{replacementproof}}

% Side Indented Theorems - https://tex.stackexchange.com/questions/429339/shifting-newtheorem
\newtheoremstyle{side}{}{}{\advance\leftskip3cm\relax\itshape\normalfont}{-4pt}
{\bfseries}{}{0pt}{
\makebox[0pt][r]{
  \smash{\parbox[t]{2.5cm}{\raggedright\thmname{#1}.
  \thmnote{\newline(#3)}}}
  \hspace{10.1pt}}}

\theoremstyle{side}
\newtheorem{note}{Note}
\newtheorem{intuition}{Intuition}
\newtheorem{claim}{Claim}
\theoremstyle{definition}

%pulls lecture files
\newcommand{\lec}[2]{%
	\foreach \c in {#1,...,#2}{%
		\IfFileExists{Lectures/lec_\c.tex} {%
			\input{Lectures/lec_\c.tex}%
		}{}%
	}%
}
%to use, in the same file directory as your header.tex and master.tex files, create a folder titled "Lectures" and put your
%lectures into their, named lec_1.tex, lec_2.tex, and so on. In the master.tex file, write \lec{a}{b} where a is the lowest
%number you want to call, and b is the highest.

%fancy headers
\usepackage{fancyhdr}
\pagestyle{fancy}
\fancyhead{}\fancyfoot{}
\fancyfoot[R]{\thepage}
\fancyfoot[C]{\leftmark}


%figures, taken from (https://castel.dev/post/lecture-notes-2)
\usepackage{import}
\usepackage{xifthen}
\usepackage{pdfpages}
\usepackage{transparent}
\newcommand{\incfig}[1]{%
    \def\svgwidth{\columnwidth}
    \import{./figures/}{#1.pdf_tex}
}

%lectures, taken from (https://castel.dev/post/lecture-notes-3)
\makeatother
\def\@lecture{}%
\newcommand{\lecture}[3]{%
	\ifthenelse{\isempty{#3}}{%

		\def\@lecture{Lecture #1}%
	}{%
		\def\@lecture{Lecture #1: #3}
	}
	\subsection*{\@lecture}
	\hfill{\small\textsf{#2}}\par
}
\makeatletter

\author{Grant Talbert}

\usepackage{titlepageBU}
\title{Modern Algebra 1}
\date{09/12/24}
\courseID{MA 541}
\professor{Dr. Duque-Rosero}
\courseSection{A1}
\newcounter{prob}\setcounter{prob}{0}
\renewcommand{\theprob}{\textbf{Problem \arabic{prob}. }}
\newcommand{\problem}{\stepcounter{prob}\noindent\theprob}
\begin{document}
\makeproblem
\problem hi

\problem Let $n\geq 2$. Define $R$ as the rotation of the regular $n$-gon by $360$ degrees, $S$ as any reflection of the $n$-gon, and $R_0$ as the identity transformation (rotation by 0 degrees). Show
\[
	D_n =\left\{ R_0,R^1,R^2, \dots, R^{n-1} ,S,RS,R^2S ..,R^{n-1}S \right\}
.\]
Note that $R^iS=SR^P{-i}$ for all $i$.
\begin{solution}
	Recall the description of the elements of $D_n$ from part (a) of problem 0. The fact that $\left\{ R_0,R^1,R^2,\ldots,R^{n-1} \right\} \subseteq D_n$ is explained precisely in part (a) of problem 1. It remains to be shown that each reflection $S_1, \dots, S_n $ has the representation $R^iS$ for some $i$. I have no idea how to show this rigorously, but the problem says an explanation will suffice. Because the $n$-gon is rigid, it has only two distinct orderings for its angles, which can be called face-up and face-down.
\end{solution}

\problem For this problem, recall that $D_6$ is the dihedral group of order 12, the group of symmetries of the hexagon.
\begin{itemize}
	\item Find elements  $A,B\in D_6$ such that $AB\neq BA$.
	\item Find elements $A,B,C\in D_6$ such that $AB=BC$ but $A\neq C$.
\end{itemize}
\begin{solution}
	For simplicity, consider the visualization below.\\
	\begin{center}
\begin{tikzpicture}
   \newdimen\R
   \R=1.5cm
   \draw (0:\R) \foreach \x in {60,120,...,360} {  -- (\x:\R) };
   \foreach \x/\l/\p in
     { 60/{$a$}/above,
      120/{$f$}/above,
      180/{$e$}/left,
      240/{$d$}/below,
      300/{$c$}/below,
      360/{$b$}/right
     }
     \node[inner sep=1pt,circle,draw,fill,label={\p:\l}] at (\x:\R) {};
\end{tikzpicture}\end{center}
Take $S$ to be the reflection fixing points $a$ and $d$.
	\begin{center}
\begin{tikzpicture}
   \newdimen\R
   \R=1.5cm
   \draw (0:\R) \foreach \x in {60,120,...,360} {  -- (\x:\R) };
   \foreach \x/\l/\p in
     { 60/{$a$}/above,
      120/{$f$}/above,
      180/{$e$}/left,
      240/{$d$}/below,
      300/{$c$}/below,
      360/{$b$}/right
     }
     \node[inner sep=1pt,circle,draw,fill,label={\p:\l}] at (\x:\R) {};
	\draw[dashed,thick] (-1,-1.75) -- (1,1.75); 
\end{tikzpicture}
\begin{tikzpicture}
	\draw [-{Stealth[length=5mm]}] (0,0) -- node[below=1.6cm] {} (2,0);
	\node[] at (0.9, 0.3) (a) {$S$};
\end{tikzpicture}
\begin{tikzpicture}
   \newdimen\R
   \R=1.5cm
   \draw (0:\R) \foreach \x in {60,120,...,360} {  -- (\x:\R) };
   \foreach \x/\l/\p in
     { 60/{$a$}/above,
      120/{$b$}/above,
      180/{$c$}/left,
      240/{$d$}/below,
      300/{$e$}/below,
      360/{$f$}/right
     }
     \node[inner sep=1pt,circle,draw,fill,label={\p:\l}] at (\x:\R) {};
	\draw[dashed,thick] (-1,-1.75) -- (1,1.75); 
\end{tikzpicture}\end{center}
From the visualization, we know $S$ maps $b$ to $f$, $e$ to $c$, and vice versa. Now take $R$ to be a clockwise rotation by $60^\circ $.
	\begin{center}
\begin{tikzpicture}
   \newdimen\R
   \R=1.5cm
   \draw (0:\R) \foreach \x in {60,120,...,360} {  -- (\x:\R) };
   \foreach \x/\l/\p in
     { 60/{$a$}/above,
      120/{$f$}/above,
      180/{$e$}/left,
      240/{$d$}/below,
      300/{$c$}/below,
      360/{$b$}/right
     }
     \node[inner sep=1pt,circle,draw,fill,label={\p:\l}] at (\x:\R) {};
     \draw [->] (1,1.6) to [out=-30,in=100] (1.8,0.3);
\end{tikzpicture}
\begin{tikzpicture}
	\draw [-{Stealth[length=5mm]}] (0,0) -- node[below=1.6cm] {} (2,0);
	\node[] at (0.9, 0.3) (a) {$R$};
\end{tikzpicture}
\begin{tikzpicture}
   \newdimen\R
   \R=1.5cm
   \draw (0:\R) \foreach \x in {60,120,...,360} {  -- (\x:\R) };
   \foreach \x/\l/\p in
     { 60/{$f$}/above,
      120/{$e$}/above,
      180/{$d$}/left,
      240/{$c$}/below,
      300/{$b$}/below,
      360/{$a$}/right
     }
     \node[inner sep=1pt,circle,draw,fill,label={\p:\l}] at (\x:\R) {};
     \draw [->] (1,1.6) to [out=-30,in=100] (1.8,0.3);
\end{tikzpicture}\end{center}
Consider the symmetries $A\coloneqq SR$ and $B\coloneqq R$. We have $AB=SRR$ and $BA=RSR$. Function composition is applied from right to left, so we apply the rightmost transformation first. Rather than tediously explain what point maps to what position, a visual proof has been given.
	\begin{center}
		$BA=RSR$\\
	\begin{tikzpicture}
   \newdimen\R
   \R=1.5cm
   \draw (0:\R) \foreach \x in {60,120,...,360} {  -- (\x:\R) };
   \foreach \x/\l/\p in
     { 60/{$a$}/above,
      120/{$f$}/above,
      180/{$e$}/left,
      240/{$d$}/below,
      300/{$c$}/below,
      360/{$b$}/right
     }
     \node[inner sep=1pt,circle,draw,fill,label={\p:\l}] at (\x:\R) {};
\end{tikzpicture}
\begin{tikzpicture}
	\draw [-{Stealth[length=5mm]}] (0,0) -- node[below=1.6cm] {} (2,0);
	\node[] at (0.9, 0.3) (a) {$R$};
\end{tikzpicture}
	\begin{tikzpicture}
   \newdimen\R
   \R=1.5cm
   \draw (0:\R) \foreach \x in {60,120,...,360} {  -- (\x:\R) };
   \foreach \x/\l/\p in
     { 60/{$f$}/above,
      120/{$e$}/above,
      180/{$d$}/left,
      240/{$c$}/below,
      300/{$b$}/below,
      360/{$a$}/right
     }
     \node[inner sep=1pt,circle,draw,fill,label={\p:\l}] at (\x:\R) {};
\end{tikzpicture}
\begin{tikzpicture}
	\draw [-{Stealth[length=5mm]}] (0,0) -- node[below=1.6cm] {} (2,0);
	\node[] at (0.9, 0.3) (a) {$S$};
\end{tikzpicture}
	\begin{tikzpicture}
   \newdimen\R
   \R=1.5cm
   \draw (0:\R) \foreach \x in {60,120,...,360} {  -- (\x:\R) };
   \foreach \x/\l/\p in
     { 60/{$f$}/above,
      120/{$a$}/above,
      180/{$b$}/left,
      240/{$c$}/below,
      300/{$d$}/below,
      360/{$e$}/right
     }
     \node[inner sep=1pt,circle,draw,fill,label={\p:\l}] at (\x:\R) {};
\end{tikzpicture}
\begin{tikzpicture}
	\draw [-{Stealth[length=5mm]}] (0,0) -- node[below=1.6cm] {} (2,0);
	\node[] at (0.9, 0.3) (a) {$R$};
\end{tikzpicture}
	\begin{tikzpicture}
   \newdimen\R
   \R=1.5cm
   \draw (0:\R) \foreach \x in {60,120,...,360} {  -- (\x:\R) };
   \foreach \x/\l/\p in
     { 60/{$a$}/above,
      120/{$b$}/above,
      180/{$c$}/left,
      240/{$d$}/below,
      300/{$e$}/below,
      360/{$f$}/right
     }
     \node[inner sep=1pt,circle,draw,fill,label={\p:\l}] at (\x:\R) {};
\end{tikzpicture}
\\$AB=SRR$\\
	\begin{tikzpicture}
   \newdimen\R
   \R=1.5cm
   \draw (0:\R) \foreach \x in {60,120,...,360} {  -- (\x:\R) };
   \foreach \x/\l/\p in
     { 60/{$a$}/above,
      120/{$f$}/above,
      180/{$e$}/left,
      240/{$d$}/below,
      300/{$c$}/below,
      360/{$b$}/right
     }
     \node[inner sep=1pt,circle,draw,fill,label={\p:\l}] at (\x:\R) {};
\end{tikzpicture}
\begin{tikzpicture}
	\draw [-{Stealth[length=5mm]}] (0,0) -- node[below=1.6cm] {} (2,0);
	\node[] at (0.9, 0.3) (a) {$RR$};
\end{tikzpicture}
	\begin{tikzpicture}
   \newdimen\R
   \R=1.5cm
   \draw (0:\R) \foreach \x in {60,120,...,360} {  -- (\x:\R) };
   \foreach \x/\l/\p in
     { 60/{$e$}/above,
      120/{$d$}/above,
      180/{$c$}/left,
      240/{$b$}/below,
      300/{$a$}/below,
      360/{$f$}/right
     }
     \node[inner sep=1pt,circle,draw,fill,label={\p:\l}] at (\x:\R) {};
\end{tikzpicture}
\begin{tikzpicture}
	\draw [-{Stealth[length=5mm]}] (0,0) -- node[below=1.6cm] {} (2,0);
	\node[] at (0.9, 0.3) (a) {$S$};
\end{tikzpicture}
	\begin{tikzpicture}
   \newdimen\R
   \R=1.5cm
   \draw (0:\R) \foreach \x in {60,120,...,360} {  -- (\x:\R) };
   \foreach \x/\l/\p in
     { 60/{$e$}/above,
      120/{$f$}/above,
      180/{$a$}/left,
      240/{$b$}/below,
      300/{$c$}/below,
      360/{$d$}/right
     }
     \node[inner sep=1pt,circle,draw,fill,label={\p:\l}] at (\x:\R) {};
\end{tikzpicture}
\end{center}
Clearly these hexagons are not in the same orientation, and as such $AB\neq BA$.

For the second part of the problem, let $A$ and $B$ be as previously defined. Notice that the symmetry $AB$ is equivalent to reflecting once and then rotating 4 times. Define $C\coloneqq RRRS$. With this definition, $BC=RRRRS$, one reflection and four rotations. Thus, $AB=BC$. However, it's not necessarily implied that $A\neq C$, since $SR$ and $RRRS$ may simply be different representations of the same symmetry. The result of the transformation $A$ can be seen as the result of the first two transformations present in $BA$. For simplicity, it has been redrawn below.
\begin{center}\begin{tikzpicture}
   \newdimen\R
   \R=1.5cm
   \draw (0:\R) \foreach \x in {60,120,...,360} {  -- (\x:\R) };
   \foreach \x/\l/\p in
     { 60/{$a$}/above,
      120/{$f$}/above,
      180/{$e$}/left,
      240/{$d$}/below,
      300/{$c$}/below,
      360/{$b$}/right
     }
     \node[inner sep=1pt,circle,draw,fill,label={\p:\l}] at (\x:\R) {};
\end{tikzpicture}
\begin{tikzpicture}
	\draw [-{Stealth[length=5mm]}] (0,0) -- node[below=1.6cm] {} (2,0);
	\node[] at (0.9, 0.3) (a) {$A$};
\end{tikzpicture}
	\begin{tikzpicture}
   \newdimen\R
   \R=1.5cm
   \draw (0:\R) \foreach \x in {60,120,...,360} {  -- (\x:\R) };
   \foreach \x/\l/\p in
     { 60/{$f$}/above,
      120/{$a$}/above,
      180/{$b$}/left,
      240/{$c$}/below,
      300/{$d$}/below,
      360/{$e$}/right
     }
     \node[inner sep=1pt,circle,draw,fill,label={\p:\l}] at (\x:\R) {};
\end{tikzpicture}\end{center}
Now consider the transformation $C $:
	\begin{center}
\begin{tikzpicture}
   \newdimen\R
   \R=1.5cm
   \draw (0:\R) \foreach \x in {60,120,...,360} {  -- (\x:\R) };
   \foreach \x/\l/\p in
     { 60/{$a$}/above,
      120/{$f$}/above,
      180/{$e$}/left,
      240/{$d$}/below,
      300/{$c$}/below,
      360/{$b$}/right
     }
     \node[inner sep=1pt,circle,draw,fill,label={\p:\l}] at (\x:\R) {};
\end{tikzpicture}
\begin{tikzpicture}
	\draw [-{Stealth[length=5mm]}] (0,0) -- node[below=1.6cm] {} (2,0);
	\node[] at (0.9, 0.3) (a) {$S$};
\end{tikzpicture}
\begin{tikzpicture}
   \newdimen\R
   \R=1.5cm
   \draw (0:\R) \foreach \x in {60,120,...,360} {  -- (\x:\R) };
   \foreach \x/\l/\p in
     { 60/{$a$}/above,
      120/{$b$}/above,
      180/{$c$}/left,
      240/{$d$}/below,
      300/{$e$}/below,
      360/{$f$}/right
     }
     \node[inner sep=1pt,circle,draw,fill,label={\p:\l}] at (\x:\R) {};
\end{tikzpicture}
\begin{tikzpicture}
	\draw [-{Stealth[length=5mm]}] (0,0) -- node[below=1.6cm] {} (2,0);
	\node[] at (0.9, 0.3) (a) {$RRR$};
\end{tikzpicture}
\begin{tikzpicture}
   \newdimen\R
   \R=1.5cm
   \draw (0:\R) \foreach \x in {60,120,...,360} {  -- (\x:\R) };
   \foreach \x/\l/\p in
     { 60/{$d$}/above,
      120/{$e$}/above,
      180/{$f$}/left,
      240/{$a$}/below,
      300/{$b$}/below,
      360/{$c$}/right
     }
     \node[inner sep=1pt,circle,draw,fill,label={\p:\l}] at (\x:\R) {};
\end{tikzpicture}
\end{center}
Clearly, $A\neq C$. As such, we have an example of $AB=BC$ for $A\neq C$.
\end{solution}

\problem Let $G=\left\{ 1,2,3,4 \right\} $ with binary operation given by multiplication modulo $n$. Show that $G$ is a group under this operation.
\begin{solution}
	Clearly ``multiplication modulo $n$'' is a typo, as the value of $n$ matters, and for $n\geq 6$, there are sitiations where for some $a,b\in G$, $ab\mod n\notin G$. For example, take $n=12$. $3,2\in G$, but $3\cdot 2=6\mod 12\notin G$. Since the problem requires $G$ is a group, we take $n=5$.

	Due to the ambiguity in the definition of the binary operation, I have chosen to show multiplication modulo $5$ is a binary operation on $G$ by brute force in the following Cayley table.
	\begin{center}
		\begin{tabular}{c|cccc}
			$\cdot $ &1&2&3&4\\
			\hline
			1&1&2&3&4\\
			2&2&4&1&3\\
			3&3&1&4&2\\
			4&4&3&1&1
		\end{tabular}
	\end{center}
	From this table, it can also be seen that $1$ is an identity element in $G$, so there exists an identity element in $G$. It can also be seen that every element has an inverse; 1 is it's own inverse since $1\cdot 1=1\mod 5$, but also $2\cdot 3=1\mod 5$, $3\cdot2 =1\mod 5$, and $4\cdot 4=1\mod 5$. So under multiplication modulo $5$, the inverse of $2$ is $3$, the inverse of $3$ is 2, and the inverse of 4 is 4, all of which are elements of $G$. Finally, we show associativity. We know that associativity holds in $\mathbb{Z}$, and $\left\{ 1,2,3,4 \right\} \subseteq \mathbb{Z}$. Thus, associativity holds for all elements of $G$, since all elements of $G$ are also integers, which are known to be associative. Therefore, $G$ is a group under the operation of multiplication modulo $5$.
\end{solution}

\problem hi
\end{document}
