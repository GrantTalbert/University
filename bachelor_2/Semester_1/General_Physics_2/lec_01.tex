\lecture{1}{Tue 03 Sep 2024 17:01}{Syllabus Day}
James Miller has no idea how to operate technology.\\
email: miller@bu.edu\\

Experiments this mf is doing/did:\\
Muon 2-g\\
Me2e

\chapter{Introduction to Electricity and Magnetism}
\begin{prev}
	Let $\mathbf{a},\mathbf{b}\in\mathbb{R}^2$ be vectors with
	\[
		\mathbf{a}=\left<a_x,a_y \right>,\quad \mathbf{b}= \left<b_x,b_y \right>
	.\]
	Then
	\[
		\mathbf{a}+\mathbf{b}=\left<a_x+b_x,a_y+b_y \right>
	.\]

	For some vector $\mathbf{R}$ which forms angle $\phi $ with the horizobntal,
	\[
		\tan \phi =\frac{R_y}{R_x}
	.\]
	Also,
	$F_g = -G\frac{m_1 m_2}{r^2}.$
\end{prev}

Electric charge is a concerved quantity, and there are two types: ``+'' and ``-''.  There are also conductors, which are materials that allow for free flow of electric charge, adn insulators which do not allow free movement of charges.

Conduction - charging method via touching

Induction - charges by:
\begin{enumerate}
	\item Bring a charged object close to a neutral object, positive and negative charges rearrange within neutral object
	\item Neutral object gruounnded, excess charges flow
	\item Remove the ground, sphere is charged
\end{enumerate}

\begin{theorem}[Coulomb's Law]\label{thm:1}
	The magnitude of the electric force between two point charges at $\mathbf{q}_1,\mathbf{q}_2$ with magnitudes $q_1,q_2$ is given as
	\[
		\mathbf{F}_{12} = k\frac{q_1 q_2}{r_{12}^2}\mathbf{\hat{r}}_{12}
	,\]
	where
	\[
		k=\frac{1}{4\pi \epsilon_0}\approx 9\times 10^9 \frac{Nm^2}{C^2}
	,\]
	\[
		r_{12}= \left\| \mathbf{q}_2 - \mathbf{q}_2 \right\| 
	,\]
	and $\mathbf{\hat{r}}$ is a unit vector specifying the direction of the force.
\end{theorem}
To find $\mathbf{\hat{r}}$, we have
\[
	\mathbf{\hat{r}}=\cos \theta \hati+\sin \theta \hatj
.\]
Unit vectors:
\[
	\mathbf{\hat{x}}\equiv \frac{\mathbf{x}}{\left\| \mathbf{x} \right\| }
.\]

Electron charge is not continuous, it's quantized. Every electron has the charge
\[
	e=1.6\times 10^{-19}C
.\]
This is the lowest charge seen on a \emph{free} particle (quarks only exist within hadrons). So the charge of an object with $n$ electrons is $q=ne$.
